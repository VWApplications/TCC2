\section{Verificação e Validação}

Essa subseção fala um pouco do processo de verificação e validação e está dividido em: Técnicas estáticas de verificação e validação, técnicas dinâmicas de verificação e validação e testes automatizados.

De acordo com \cite{pressman} a engenharia de software é um corpo de conhecimento que aplica os princípios de engenharia com o objetivo de produzir software de alta qualidade e baixo custo. Porém no processo de desenvolvimento erros no produto ainda podem ocorrer.

Várias atividades sobre a perspectiva da garantia da qualidade têm sido introduzida ao longo do processo de desenvolvimento, entre elas temos as atividades de verificação e validação, com o objetivo de minimizar a ocorrência desses erros \cite{jino}.

Propósito do processo de verificação é confirmar que cada serviço e/ou produto de trabalho do processo ou do projeto atende apropriadamente os requisitos especificados \cite{pressman1}.

O processo de verificação é detalhado pelo guia, segundo o \cite{mpsbr}, em 6 tarefas gerais:

\begin{itemize}
  \item \textbf{VER 1}: Identificar os produtos de trabalho a serem verificados.
  \item \textbf{VER 2}: Desenvolver e implementar uma estratégia de verificação, com definição de um cronograma,
    revisores envolvidos, métodos para verificação e qualquer material a ser utilizado na verificação.
  \item \textbf{VER 3}: Identificar critérios e procedimentos para verificação dos produtos de trabalho,
    além da definição do ambiente de verificação.
  \item \textbf{VER 4}: Executar revisão por pares e testes e atividades de verificação.
  \item \textbf{VER 5}: Identificar e registrar defeitos.
  \item \textbf{VER 6}: Analisar os resultados gerados e encaminhar aos envolvidos.
\end{itemize}

O objetivo da validação é validar que um produto de software atenderá a seu objetivo quando colocado no ambiente para o qual foi desenvolvido \cite{sommerville}.

De forma geral o processo de validação tem seu foco em como avaliar a qualidade de um produto ou componente de produto, assegurando que os objetivos e ou necessidades dos clientes sejam atendidas quando colocado em seu ambiente de produção, ou seja, o objetivo da validação é garantir que o produto correto está sendo desenvolvido \cite{mpsbr}.

O processo de validação é detalhado pelo guia, segundo o \cite{mpsbr}, em 7 tarefas gerais, como resultados esperados:

\begin{itemize}
  \item \textbf{VAL 1}: Identificar os produtos de trabalho a serem validados.
  \item \textbf{VAL 2}: Desenvolver e implementar uma estratégia de validação, com definição de um cronograma,
    revisores envolvidos, métodos para validação e qualquer material a ser utilizado na validação.
  \item \textbf{VAL 3}: Identificar critérios e procedimentos para validação dos produtos de trabalho, além da
    definição do ambiente de validação.
  \item \textbf{VAL 4}: Executar atividades de validação para garantir que o produto esteja pronto para ser
    disponibilizado em ambiente de uso.
  \item \textbf{VAL 5}: Identificar e registrar problemas.
  \item \textbf{VAL 6}: Analisar os resultados obtidos e encaminhar aos envolvidos.
  \item \textbf{VAL 7}: Obter evidências de que os produtos de software desenvolvidos estão prontos para o uso
    pretendido são fornecidas.
\end{itemize}

A atividade de verificação e validação apresenta dois tipos de técnicas que podem ser aplicadas no processo de desenvolvimento de software. São elas: técnicas estáticas e técnicas dinâmicas de verificação e validação de software \cite{myers}.

\subsection{Técnicas estáticas}

De acordo com \cite{myers} Como técnicas estáticas de verificação e validação temos:

\begin{itemize}
  \item \textbf{Inspeção}: A inspeção é um processo de revisão formal de software e corresponde a uma das mais importantes atividades de Garantia de Qualidade de Software, o seu principal objetivo é a descoberta antecipada de falhas e é dividido em 6 partes: planejamento, apresentação, preparação, reunião de inspeção, retrabalho e acompanhamento.
  \item \textbf{Walkthrough}: Nesta técnica a revisão é feito por meio da execução passo a passo de um procedimento ou programa, porém realizado no papel. O principal objetivo é encontrar erros e envolve equipes pequenas de três a cinco pessoas, na qual é feito uma simulação da execução por cada revisor por meio de um conjunto de casos de testes disponibilizado pelo testador.
  \item \textbf{Peer-Review}: É uma técnica realizada em pares de programadores com mesmo nível de conhecimento. O objetivo desta técnica é obter pontos de vista diferentes dos desenvolvedores a fim de encontrar problemas de qualidade. Deve ser analisado o produto e não o desenvolvedor.
\end{itemize}

\subsection{Técnicas dinâmicas}

A atividade de testes dentro das técnicas dinâmicas de verificação e validação são as mais utilizadas, essas fornecem evidências da confiabilidade do software em complemento a outras atividades, como o uso técnicas estáticas de verificação e validação \cite{jino}.

Ela consiste na análise dinâmica do produto e tem como objetivo encontrar falhas nesse produto, para um possível processo de depuração e, por consequência, o aumento da confiança de que ele esteja correto \cite{rocha}.

Sobre casos de testes temos alguns conceitos definidos, como: níveis de testes, tipos de testes e técnicas de testes.

\subsubsection{Níveis de teste}

O teste de produto de software concretiza-se em 3 fases de testes: De unidade, integração e sistema \cite{myers}.

Os testes de unidade cada unidade do programa é testada, isolada das demais unidades. Esse teste, verifica se a unidade funciona de forma adequada aos tipos de entrada esperados. Normalmente na orientação a objeto são as classes ou modelos.  Ele a testa de maneira isolada geralmente simulando as prováveis dependências que aquela unidade tem.

Quando todas as unidades já tiverem sido testadas, a próxima fase é realizar o teste de integração, para assegurar que as interfaces entre as unidades foram definidas e tratadas adequadamente. É aquele que testa a integração entre duas partes do seu sistema. Os testes das controladora, por exemplo, onde seu teste vai até o banco de dados, é um teste de integração. Afinal, você está testando a integração do seu sistema com o sistema externo, que é o banco de dados. Testes que garantem que suas classes comunicam-se bem com serviços web, escrevem arquivos texto, ou mesmo mandam mensagens via socket são considerados testes de integração.

Funcionamento do sistema como um todo, com todas as unidades trabalhando juntas. De acordo com o MPS.BR o teste do sistema envolve: teste funcional que verifica se o sistema integrado realiza as funções especificadas nos requisitos; teste de desempenho que avalia como o sistema se comporta em relação aos requisitos não-funcionais especificados, tais como tempo de resposta, uso do processador, segurança, dentre outros; teste de aceitação que verificar a iteração de um usuário com o software, documentação do usuário e treinamento; e teste de instalação que são testes de scripts de instalação para verificar se o software é instalado sem nenhum problema no host dos clientes.

\subsubsection{Tipos de testes}

De acordo com \cite{myers} temos dois tipos de testes: teste funcional, ou seja, teste baseado em requisitos funcionais e teste não funcional, teste baseado em requisitos não funcionais, por exemplo, qualidade de código.

\subsubsection{Técnicas de testes}

Técnica é o processo que vai assegurar perfeito funcionamento de alguns aspectos de software ou de sua unidade. De
acordo com \cite{myers} existem quatro tipos de técnicas principais, são elas:

\begin{itemize}
  \item \textbf{Caixa Preta}: Aborda o software sem se preocupar com a forma como ele foi implementado, ou seja, aborda o software de um ponto de vista macroscópico. Esse teste é baseado na analise funcional do software ele garante que os requisitos funcionem conforme o especificado, é inseridos alguns dados e espera-se na saída o resultado de como foi projetado os requisitos. Essa técnica pode ser aplicada em todos os níves de teste citados.
  \item \textbf{Caixa Branca}: Essa técnica é o oposto da caixa preta, já que estabelece os requisitos de teste com base na implementação do código. Esse teste tem por objetivo testar o código fonte cobrindo as funcionalidades do componente de software, ele testa cada linha de código possível, testar os fluxos básicos e os alternativos.
  \item \textbf{Particionamento de Equivalência}: É uma técnica caixa preta que agrupa e otimiza casos de testes, afim de fazer a maior cobertura possível do sistema. Ela propõe a separação das possíveis entradas em categorias diferentes. O objetivo dessa técnica é eliminar os casos de testes redundante, por exemplo, valores entre 1990 e 2000, podemos pegar um único representante para todos esses casos, 1993 por exemplo e pegamos representantes para dados invalidos, como dados negativos ou fora do intervalor proposto, por exemplo, -10, 1980, 2001, entre outros. Os casos de teste devem ser construídos a partir das partições criadas.
  \item \textbf{Análise de Valor Limite}: Também é uma técnica caixa preta que visa identificar o comportamento nos limites de uma partição de equivalência, ou seja, seus máximus e mínimus, que é onde existe maior probabilidade de estar incorreto. Análise do valor limite pode ser aplicada em todos os níveis de teste. Os limites são áreas onde testes estão mais propensos a indicar defeitos. Por exemplo, os valores limites de 1900 a 2004 são 1899, 1900, 2004, 2005. Ela complementa o particionamento de equivalência.
\end{itemize}

\subsubsection{Testes automatizados}

Para \cite{kon} no mundo ideal, os sistemas devem não só fazer corretamente o que o cliente precisa, mas fazê-lo de forma segura, eficiente e escalável, além de ser de fácil manutenção, evolução e flexíveis.

Hoje em dia essa característica são asseguradas através de testes manuais. Após os três passos convencionais para o desenvolvimento de uma funcionalidade que são: estudar o problema, buscar a solução e implementá-la o desenvolvedor faz testes manuais para verificar se está tudo funcionando como o esperado. Os desenvolvedores precisam encontrar o erro, corrigi-lo e refazer o conjunto de testes manuais. Esses testes podem ser executados por desenvolvedores, usuários ou mesmo por equipes especializadas em testes \cite{kon}.

De acordo com \cite{kon} a dificuldade da realização de testes manuais leva à aparição de diversos problemas, tais como atraso nas entregas, produtos com grande quantidade de erros, dificuldade de manutenção e evolução. A execução de um caso de teste é rápida e efetiva, porém a repetição de um vasto conjunto de testes manualmente é uma tarefa cansativa e desgastante e isso leva ao cenário mais conhecido como “bola de neve”, levando a erros de regressão, ou seja, erros em módulos do sistema que estavam funcionando corretamente e deixaram de funcionar. A tendência é este ciclo se repetir até que a manutenção do sistema se torne uma tarefa tão custosa que passa a valer a pena refazê-lo do zero.

E aí que entra metodologias como o XP que recomendam que todos controlem a qualidade do produto todos os dias e a todo momento. Pois acreditam que prevenir defeitos é mais fácil e barato que identificá-los e corrigi-los. E essa em particular recomenda testes automatizados para ajudar a garantir a qualidade do mesmo \cite{kon}.

Testes automatizados são scripts que exercitam funcionalidades do sistema sendo testado e fazem verificações automáticas nos efeitos colaterais obtidos, ou seja, ao realizar o teste de regressão irá identificar se há alguma funcionalidade já implementada que parou de funcionar. Todos os casos de testes podem ser facilmente repetidos a qualquer momento e com pouco esforço e é possível criar situações de testes bem elaboradas e complexas do que as realizadas manualmente, por exemplo, simular centenas ou milhares de usuários acessando um sistema ao mesmo tempo \cite{kon}.

\subsection{Fechamento da seção}

Todas as características acima citadas ajudam a diminuir a quantidade de erros e aumentar a qualidade do software. Com isso mudanças no sistema podem ser feitos com segurança, o que aumenta a vida útil do produto e o torna mais atraente para o mercado.
