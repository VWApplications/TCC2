\section{Team Based Learning}

Nesta seção, será descrito os conceitos-chave relacionado à metodologia ativa de aprendizagem TBL e está dividido em: O que é o TBL?, Detalhando cada etapa do TBL, Preparando um módulo em TBL e os pontos positivos e negativos encontrados na aplicação do TBL em algumas universidades.

\subsection{O que é o TBL?}

De acordo com \cite{burgess} o TBL é uma estratégia educacional para grandes classes que, a partir da coordenação do professor, possibilita a interação e colaboração em pequenos grupos centrados no aluno para melhorar o processo de aprendizagem e capacitar os alunos para o mercado.

O TBL foi desenvolvido para cursos de administração nos anos de 1970, por Larry Michaelsen \cite{sweet} e posteriormente foi usado também em medicina e implementado em cursos de engenharia \cite{matalonga}

A abordagem TBL centra-se em promover o pensamento crítico de múltiplas formas, a fim de alcançar uma aprendizagem de nível superior \cite{gomez}. Outra importante característica é a aprendizagem baseada no diálogo e na interação entre os alunos, o que contempla as habilidades de comunicação e trabalho colaborativo em grupos, que será necessária ao futuro profissional \cite{bollela}.

Além disso a aprendizagem colaborativa não só  promove o engajamento e a aquisição de conhecimento pelos alunos, como também ajuda o desenvolvimento de habilidades de trabalho em grupo, como a comunicação e a colaboração, ambos importantes para a profissão de engenharia. A colaboração apresenta importantes resultados educacionais, como pensamento crítico, raciocínio moral, eficácia intercultural e bem-estar pessoal \cite{cabrera}.

O TBL pode complementar ou até mesmo substituir um curso focado em aulas expositivas \cite{dean}. Não requer múltiplas salas e nem vários docentes atuando, os alunos são induzidos a se prepararem para as atividades do TBL. Os alunos não precisam ter instruções específicas para trabalho em grupo, já que aprendem a medida que as sessões acontecem. Além de ter sua fundamentação teórica baseada no construtivismo que é uma das características das atividades colaborativas, em que o professor se torna facilitador para a aprendizagem. Os alunos tornam-se mestres de suas próprias experiências educacionais passando de um agente passivo para um agente ativo de sua aprendizagem \cite{bollela}.

Essa metodologia envolve atividades individuais e em grupos. As atividades individuais visam preparar o aluno para as atividades que se segue. Já nas atividades em grupo, estes trabalham juntos promovendo o aprendizado ativo e efetivo ao longo do semestre, resolvendo problemas, discutindo e aplicar conceitos inicialmente aprendidos através das leituras individuais. Com isso  produz maiores conquistas e relacionamentos mais saudáveis e mais positivos entre alunos, do que relacionamentos competitivos ou experiências individuais \cite{gomez}.

De acordo com \cite{sweet} o TBL é regida por quatro princípios:

\begin{enumerate}
  \item \textbf{Os grupos devem ser devidamente formados e gerenciados}: Todo o processo de aprendizagem é realizado em torno da formação e interação de indivíduos em pequenos grupos permanentes e diversificadas.
  \item \textbf{Os alunos devem ser responsáveis pelo qualidade do seu trabalho}: O TBL deve promover a noção de que os alunos devem ser responsáveis pelo seu aprendizado e pela qualidade de seu trabalho.
  \item \textbf{As atividades do grupo na classe devem promover o aprendizado e o desenvolvimento do grupo}: Para isso é necessário promover a discussão e a tomada de decisões.
  \item \textbf{Os alunos devem receber comentários frequentes e imediatos}: O feedback é essencial para o aprendizado e a retenção do conhecimento.
\end{enumerate}

Ao aderir aos quatro elementos essenciais acima, os professores criam um contexto que promove a quantidade e a qualidade da interação dos grupos, tornando-os mais ativos. Além disso, ao longo do tempo, a confiança dos alunos em seu grupo cresce até o ponto em que eles estão dispostos e capazes de enfrentar tarefas difíceis com pouca ou nenhuma ajuda externa \cite{sweet}.

A organização de uma atividade utilizando o TBL é definida nas seguintes etapas: Primeiro deve ser formado os grupos.  Estes devem ser compostos de cinco a sete integrantes e deve ser formado de forma diversificada, oferecendo os recursos necessários.

São fatores dificultadores à coesão do grupo: vínculos afetivos entre os membros, expertise diferenciada, entre outros.  A tarefa de formação dos grupos devem ser realizadas somente pelo professor. Após isso deve ser realizada as etapas propostas pelo TBL \cite{bollela}.

Cada módulo, unidade de trabalho ou tema coerente dentro do curso, segue um processo de aprendizagem iterativo que repete uma sequência de atividades propostas pelo TBL especificado na figura \ref{fig:tbl}:

\begin{enumerate}
  \item \textbf{Preparação individual} através de leitura fora da classe dos materiais disponibilizados e exercícios realizados.
  \item \textbf{Avaliação de prontidão ou garantia de preparo} (RAT – \textit{Readiness Assurance Test}) através de testes individuais (iRAT) e em grupo (gRAT). Nesta etapa, as atividades desenvolvidas buscam checar e garantir que o aluno está preparado e pronto para resolver testes individualmente, para contribuir com o seu grupo e aplicar os conhecimentos na etapa seguinte do TBL. Nesta etapa também é aplicado a apelação caso algum aluno se oponha ao resultado da avaliação.
  \item \textbf{Aplicação dos conhecimentos} adquiridos pelo grupo por meio da resolução de problemas reais, essa atividade deve ocupar a maior parte do tempo.
  \item \textbf{Avaliação em pares} para avaliar o desempenho de cada membro do grupo.
\end{enumerate}

\begin{figure}[h!]
	\centering
  \includegraphics[keepaspectratio=true,scale=0.3]{figuras/tbl.eps}
  \caption[Processo iterativo TBL.]{Processo iterativo TBL. Fonte: Autor, baseado em \cite{bollela}}
	\label{fig:tbl}
\end{figure}

Para \cite{bollela} um aspecto crítico para implantar o TBL é a colaboração dos alunos dentro dos grupos. É importante que os membros do grupo busquem uma boa interação para que não haja conflitos. A intervenção do professor deve ser adiada o máximo possível, permitindo que o próprio grupo busque a solução de seus problemas principalmente na etapa de aplicação de conceitos.

Ele deve desenvolver questões ou testes que exijam dos grupos uma resposta (um produto) que possa ser facilmente observada e comparada entre os outros grupos e com possibilidade de incluir a perspectiva do especialista. Por isso a aplicação de apenas algumas etapas do TBL é alvo de críticas de seus criadores \cite{sweet}. Embora customizações sejam inevitáveis, é necessário ter a clareza de que, com isto, nem todos os benéficos atributos da metodologia serão alcançados.

A etapa de aplicação do conhecimento deve ser estruturada seguindo alguns preceitos. Os quatro princípios básicos para elaborar esta fase são conhecidos em inglês como os 4 S’s: Problema significativo e real (\textit{Significant}), O mesmo problema para todas os grupos (\textit{Same}), Respostas curtas e específicas (\textit{Specific}) e Relatos simultâneos (\textit{Simultaneous report}), ou seja, as respostas devem ser mostradas simultaneamente, de modo a inibir que alguns grupos manifestem sua resposta a partir da argumentação de outros grupos \cite{sweet}.

\subsection{Detalhando cada etapa do TBL}

\subsubsection{Etapa 1: Preparação individual}

Os alunos devem ser responsáveis por se prepararem individualmente para as etapas seguintes do TBL. Se os alunos não se prepararem eles não serão capazes de contribuir para o desempenho do seu grupo. A falta desta preparação dificulta o desenvolvimento de coesão do grupo e resulta em ressentimento por parte do mesmo que se preparou \cite{sweet}.

\subsubsection{Etapa 2: Garantía de preparo}

O RAT: \textit{Readiness assurance test} (Avaliação de garantia de preparo) que deve ser realizado de maneira individual iRAT e depois em grupo gRAT. É o mecanismo básico que garante a responsabilidade individual pela preparação \cite{sweet}. É dividido em 4 atividades:

\begin{enumerate}
  \item \textbf{Avaliação iRAT}: Respondido individualmente sem consulta a qualquer material, consiste de algumas questões de múltipla escolha contemplando os conceitos mais relevantes das leituras ou das atividades indicadas previamente. cada questão irá ter quatro alternativas e o aluno poderá distribuir quatro pontos entre as alternativas da forma que desejar.
  \item \textbf{Avaliação gRAT}: Respondido em grupo sem consulta, os alunos devem discutir as mesmas questões e cada membro deve defender e argumentar as razões para sua escolha até o grupo decidir qual é a melhor resposta, exercitando suas habilidades de comunicação, argumentação e convencimento. Consiste das mesmas questões do iRAT, terá feedbacks imediatos da resposta certa.
  \item \textbf{Apelação}: Após as avaliações os grupos podem recorrer a apelação no caso de não concordar com a resposta indicada como correta, todo apelo deve ser feito acompanhado de argumentação, e com consulta a fontes bibliográficas pertinentes.
  \item \textbf{Feedback pelo professor}: Assim que as atividades acima forem concluídas o professor, buscando clarear conceitos fundamentais, oferece feedback a todos simultaneamente, de acordo com as questões que mais gerou dificuldade nos alunos. Ao final desta etapa, os alunos devem estar confiantes a respeito dos conceitos fundamentais e poderão aplicá-los para resolver problemas mais complexos que serão oferecidos na etapa de aplicação do conhecimento, que se segue no processo do TBL.
\end{enumerate}

\subsubsection{Etapa 3: Aplicação de conceitos}

Essa etapa é responsável pela aplicação do conhecimento adquiridos por meio da resolução de situações, problemas ou cenários relevantes e presentes na prática profissional diária do aluno, ou seja, problemas reais que ele pode enfrentar, preparando-os para o que o mercado cobra, deve ocupar a maior parte da carga horária. O fundamental é que todos os grupos estejam preparados para argumentar sobre a escolha que fizeram \cite{sweet}.

\subsubsection{Etapa 4: Avaliação em pares}

De acordo com \cite{bollela} os alunos são avaliados pelo seu desempenho individual e também pelo resultado do trabalho em grupo, além de se submeterem à avaliação entre os pares, o que incrementa a responsabilização. A avaliação pelos pares é essencial, pois, os componentes do grupo são, normalmente, os únicos que têm informações suficientes para avaliar com precisão a contribuição do outro.

A avaliação em pares, cada integrante do grupo avaliará os outros integrantes, será disponibilizado uma pontuação máxima para cada membro distribuir entre os outros membros e um campo para dizer o porquê da pontuação. Essa avaliação não precisa se identificar, ela é anônima e o cálculo das pontuações fará parte da nota individual de cada membro.

Conclui-se, assim, um módulo ou unidade educacional TBL.

\subsection{Preparo de um módulo em TBL}

De acordo com \cite{bollela}, três mudanças são necessárias quando se modifica a estratégia pedagógica de uma aula expositiva, centrada no professor, para uma atividade do tipo TBL, centrada no aluno:

\begin{itemize}
  \item Os objetivos primários devem ser trabalhados apenas com conceitos chaves para objetivos que incluem a compreensão de como estes conceitos devem ser aplicados em situações reais.
  \item O professor passa de alguém que oferece informações e conceitos, para alguém que facilita o aprendizado, ou seja, ele entrega a responsabilidade de gerar conhecimento para o aluno.
  \item É necessária uma mudança no papel e função dos alunos, que saem da posição de receptores passivos da informação para responsáveis pela aquisição do conhecimento e membros integrantes de um grupo que trabalha de forma colaborativa para compreender como aplicar o conteúdo na solução de problemas realísticos e contextualizados.
\end{itemize}

Segundo \cite{bollela} uma atividade inicial de treinamento usando o TBL com os alunos deve ser preparada para a primeira aproximação dos estudantes com a metodologia.

\subsection{Pontos positivos e negativos da metodologia}

Foram encontrados na literatura alguns pontos negativos sobre a implantação do TBL, entre eles temos: resistência dos alunos, pois esperavam aulas formais \cite{davis} e \cite{matalonga}, Insatisfação dos alunos devido a limitação de ferramentas online como fóruns de discussão \cite{awatramani}, queixas da demanda de leitura com alto consumo de tempo extraclasse, além de alguns alunos reclamaram da falta de comprometimento de alguns colegas do grupo \cite{ramos}.

E vários pontos positivos foram encontrados:

\begin{itemize}
  \item Em uma experiência do uso do TBL, no curso de Engenharia de Software da Faculdade do Gama, iniciada no ano de 2017 com duração de 3 semestres, evolvendo 5 disciplinas e uma média de 500 alunos. Estes relataram uma preferência pelo uso do TBL quando comparado ao uso de aulas expositivas convencionais, vários motivos se deram em relação a isso, entre eles: maior interação e troca de conhecimento, discussão de temas abordados de forma a promover maior diversidade de pontos de vista, maior fixação do conteúdo e participação mais ativa em sala de aula, feedbacks imediatos, maior interesse na aquisição do conhecimento devido a parte prática da metodologia, entre outros. E também teve vários pontos positivos especificados pelos professores \cite{ramos};
  \item Em um curso de programação de primeiro semestre houve um aumento de 50\% para 75\% da taxa de aprovação dos alunos do curso e diminuição das taxas de abandono \cite{matalonga};
  \item Em outro curso de programação o desempenho dos alunos melhoraram com uma abordagem mista, mistura de TBL com palestras formais, além da pesquisa de amostragem positiva dos alunos \cite{elnagar};
  \item Na universidade ORT no Uruguai no curso de engenharia de software os alunos tiveram uma percepção positiva de sua aprendizagem ao comparar o curso com TBL com os cursos tradicionais. Eles não acharam o estudo de todas as unidades temáticas estressantes e acreditam que o TBL atende ao seu estilo de aprendizagem individual \cite{matalonga};
  \item Em um curso introdutório de programação, os resultados mostraram que o TBL ajudou os alunos a alcançar um nível
    mais alto de compreensão em um curto período de tempo \cite{cabrera};
  \item No curso de Engenharia de informática sobre o design de sistemas embarcados a pontuação alcançada usando o TBL foi relativamente boa comparada às aulas tradicionais e em laboratório \cite{awatramani};
  \item Em um curso de desenvolvimento profissional da universidade de Oklahoma os comentários finais dos alunos foram satisfatórios, por exemplo, muitos conseguiram sair da zona de conforto, aprenderam habilidades valiosas como: trabalho em equipe, escrever documentos técnicos, apresentar os resultados da sua pesquisa e ideias e falar e argumentar em público \cite{davis}.
\end{itemize}

Com a análise da implantação do TBL em diversos cursos de universidades diferentes observamos que houve em praticamente todos os casos uma aceitação positiva por parte dos envolvidos, tornando a metodologia bastante aceita por parte dos alunos.

De acordo com \cite{ramos} o TBL prepara os alunos para o desenvolvimento de capacidades previstas no Projeto Pedagógico do curso de engenharia de software. Por exemplo, em termos de capacidade técnica temos estudos teóricos e resolução de problemas na fase de aplicação de conceitos. Capacidade escrita temos a redação de apelações quando um aluno não concorda com algo em relação à avaliação e quanto a capacidade oral os alunos apresentam argumentações ao responder as avaliações em grupo, há uma discussão para chegar-se a um consenso da resposta correta.
