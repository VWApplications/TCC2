\chapter[Introdução]{Introdução}

Neste capítulo, é introduzido o trabalho de conclusão de curso. O capítulo está dividido em seções que estão dispostas da seguinte maneira: na seção 1.1 é contextualizado de forma resumida o que são as metodologias ativas de aprendizado até chegar no foco que é a metodologia \textit{Team Based Learning}; na seção 1.2 é apresentado a problemática; na seção 1.3 é apresentado a justificativa; na seção 1.4 os objetivos; na seção 1.5 é apresentado a metodologia de desenvolvimento e por fim, na seção 1.6 é apresentado a organização do trabalho.

\section{Contextualização}

O mercado de engenharia exige algumas competências que devem ser enfatizadas, entre elas destacam-se: capacidade de
trabalho em equipe, análise de dados, resolução de problemas reais \cite{davis}. As metodologias ativas de aprendizagem
tem como objetivo preencher essas lacunas que o mercado exige dos alunos, entre elas temos o \textit{Team-Base Learning}
(TBL) que é uma metodologia de aprendizagem colaborativa e o \textit{Problem-Based Learning} (PBL) que é mais focada na resolução de problemas complexos com um toque ainda na abordagem pedagógica de ensino tradicional \cite{cabrera}.

A abordagem de aprendizado baseada em problema ou PBL é uma abordagem pedagógica dentro dos paradigmas de aprendizagem
construtiva e social, pelo qual os alunos colaboram para resolver problemas específicos e complexos com ajuda de
seminários e aulas \cite{gomez}. Já o TBL implementa o construtivismo que dá ênfase no papel do aluno como mestre de
suas próprias experiências educacionais. Eles saem de agente passivo do conhecimento para um agente ativo (Bruner, 1986,
1990, 1996; Piaget 1970; Wu, Bieber e Hiltz 2009, citado por \citeauthor{gomez}, \citeyear{gomez}).

\section{Problemática}

Embora existam ferramentas para apoiar o uso do TBL como CMCs (\textit{Computer-Mediated Communication}), a literatura não trata de ferramentas para automatizar a execução e implantação do TBL no ambiente acadêmico.

\section{Justificativa}

Procurando colaborar com essa problemática, o presente trabalho propôs a construção de uma plataforma para automatizar
todo o processo da metodologia ativa de aprendizado: \textit{Team Based Learning}.

\section{Objetivo}

O objetivo deste trabalho é a implementação de uma ferramenta chamada PGTBL para automatizar o processo de uso da metodologia ativa de aprendizado Team Based Learning (TBL) e temos como objetivos específicos:

\begin{itemize}
  \item Estudar o contexto do Team Based Learning;
  \item Especificar os requisitos;
  \item Adaptar a metodologia ágil para as necessidades da equipe e do projeto;
  \item Desenvolver a ferramenta e testá-la em um contexto real.
\end{itemize}

\section{Metodologia}

A metodologia utilizada para a implementação da ferramenta é a metodologia ágil com adaptações para a utilização do SCRUM, XP e SAFe para ser utilizado por um único desenvolvedor. Além disso a ferramenta terá acesso livre para contribuições externas durante seu desenvolvimento, aplicando todas as boas práticas definidas na comunidade \textit{Open Source}.

\section{Organização do Trabalho}

O trabalho está organizado da seguinte forma: No Capítulo 2 é apresentado o referencial teórico quanto ao contexto do TBL, software livre, gerenciamento utilizando scrum, xp e safe, qualidade de código utilizando como base o \textit{Goal Question Metric} (GQM) para coleta de métricas. No Capítulo 3 é apresentado a metodologia de desenvolvimento, ou seja, será apresentado passo a passo o processo na qual o desenvolvimento foi realizado. No Capítulo 4 é apresentado a proposta de trabalho que terá uma visão geral do produto a ser desenvolvido, além de algumas informações técnicas e de gerenciamento como Roadmap, ferramentas, riscos, EAP do projeto e requisitos iniciais. E por último temos as Considerações Finais na qual terá um resumo geral do que foi falado, e o que se pode esperar do projeto no futuro.
