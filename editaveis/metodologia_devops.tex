\subsection{Subprocesso de DevOps}

Relembrando que DevOps é a combinação de filosofias, práticas e ferramentas que aumentam a capacidade de distribuir aplicativos e serviços em alta velocidade. Essas velocidade permite que seus clientes sejam atendidos de forma melhor e as empresas conseguem competir de forma mais eficaz no mercado.

\begin{figure}[H]
	\centering
  \includegraphics[keepaspectratio=true,scale=0.4]{figuras/devops.eps}
  \caption[Subprocesso de DevOps.]{Subprocesso de DevOps. Fonte: Autor}
	\label{fig:devops}
\end{figure}

\begin{itemize}
  \item \textbf{Criação da branch para fazer a funcionalidade}:
  \begin{itemize}
    \item \textbf{Descrição}: O desenvolvedor irá criar uma nova branch a partir da devel para criar a nova funcionalidade
      proposta de acordo com a política de branch estabelecida.
    \item \textbf{Entradas}: N/A
    \item \textbf{Saídas}: Nova branch da funcionalidade.
  \end{itemize}
  \item \textbf{Construção do código}:
  \begin{itemize}
    \item \textbf{Descrição}: Com a nova branch criada o desenvolvedor irá criar o código para a nova funcionalidade.
    \item \textbf{Entradas}: Branch
    \item \textbf{Saídas}: Nova funcionalidade.
  \end{itemize}
  \item \textbf{Envio do código para o github}:
  \begin{itemize}
    \item \textbf{Descrição}: Assim que o código estiver pronto, o desenvolvedor irá enviar o código para o repositorio do
    github.
    \item \textbf{Entradas}: Nova funcionalidade
    \item \textbf{Saídas}: N/A
  \end{itemize}
  \item \textbf{Pull Request da branch atual para a branch devel}:
  \begin{itemize}
    \item \textbf{Descrição}: Assim que o código for enviado para o github o desenvolvedor irá abrir um Pull Request para
    que comece o processo de Integração Contínua e Deploy Continúo.
    \item \textbf{Entradas}: Nova funcionalidade
    \item \textbf{Saídas}: N/A
  \end{itemize}
  \item \textbf{Execução do Travis CI}:
  \begin{itemize}
    \item \textbf{Descrição}: Processo automatizado para execução da Integração Contínua e Deploy Contínuo.
    \item \textbf{Entradas}: Nova funcionalidade
    \item \textbf{Saídas}: N/A
  \end{itemize}
  \item \textbf{Testes e Qualidade de Código}:
  \begin{itemize}
    \item \textbf{Descrição}: Na integração Contínua de forma automatizada será executado os testes unitários, integração
    e aceitação, além de verificar a qualidade do código por meio da ferramenta de análise estática.
    \item \textbf{Entradas}: Nova funcionalidade
    \item \textbf{Saídas}: Relatório com a cobertura de teste e qualidade de código.
  \end{itemize}
  \item \textbf{Aprovação e Merge do Pull Request}:
  \begin{itemize}
    \item \textbf{Descrição}: Assim que a integração contínua for finalizada será feito o merge do código para a branch
    devel ou master e começará o deploy contínuo do software.
    \item \textbf{Entradas}: Nova funcionalidade
    \item \textbf{Saídas}: Versão publicavel do software.
  \end{itemize}
  \item \textbf{Publicação das imagens no Dockerhub}:
  \begin{itemize}
    \item \textbf{Descrição}: Por meio de Scripts automatizados o deploy continuo irá mandar a imagem do docker de
    homologação ou produção gerada para o respositório de imagens chamado Dockerhub.
    \item \textbf{Entradas}: Imagem do software.
    \item \textbf{Saídas}: Versão publicavel da imagem do software de homologação ou produção.
  \end{itemize}
  \item \textbf{Conexão na máquina de deploy}:
  \begin{itemize}
    \item \textbf{Descrição}: Por meio de SSH o script irá entrar na máquina de homologação ou produção
    \item \textbf{Entradas}: Scrips de conexão.
    \item \textbf{Saídas}: Conexão na máquina de homologação ou produção.
  \end{itemize}
  \item \textbf{Atualização dos ambientes com a última versão dos containers}:
  \begin{itemize}
    \item \textbf{Descrição}: Assim que tiver dentro da máquina o script irá atualizar o respositório com as novas
    modificações e irá subir o software por meio das imagens armazenadas no dockerhub.
    \item \textbf{Entradas}: Scrips de conexão e Imagens do Dockerhub.
    \item \textbf{Saídas}: Software em produção ou Homologação.
  \end{itemize}
  \item \textbf{Disponibilidade da nova versão para o usuãrio}:
  \begin{itemize}
    \item \textbf{Descrição}: Assim que subir a imagem a nova versão do software estará pronta para uso.
    \item \textbf{Entradas}: Scrips de conexão e Imagens do Dockerhub.
    \item \textbf{Saídas}: Software em produção ou Homologação.
  \end{itemize}
  \item \textbf{Pull Request da branch devel para a branch master}:
  \begin{itemize}
    \item \textbf{Descrição}: Se o processo acima foi dentro da branch devel, ou seja, ambiente de homologação, o
    desenvolvedor terá que mandar um Pull Request para a branch master para subir o ambiente de produção.
    \item \textbf{Entradas}: Novas Funcionalidade.
    \item \textbf{Saídas}: Software em produção.
  \end{itemize}
\end{itemize}
