\chapter{Considerações finais}

Este trabalho abordou uma metodologia nova que vem sendo bastante empregado em várias instituições de ensino para preparar melhor os alunos para o mercado de trabalho chamada \textit{Team Based Learning}. Entretanto, não foram encontrados ferramentas que automatizam essa metodologia, apenas ferramentas que complementa a implantação da mesma.

Com o intuito de fechar essa lacuna em relação a ferramenta foi proposto um software chamado PGTBL que irá automatizar todo o processo da metodologia para que o professor possa apenas focar no conteúdo e facilitar o gerenciamento da turma, não tendo que gastar recursos para poder implantar a metodologia em sala de aula.

O \textit{software} foi pensado sempre com foco na qualidade do mesmo, e em aspectos de engenharia de software como engenharia de requisitos, UX, devops, métodos ágeis, medição e análise, testes automatizados, integração e entrega contínua, boas práticas de programação entre outros.

Foi proposto um escopo bem fechado em relação a ferramenta nessas duas entregas planejadas, porém, para evoluções futuras após esse trabalho, aplicação poderá ter um sistema de \textit{machine learning} e gamificação na parte de preparação do TBL além de ser extensível a qualquer disciplina ou curso tanto superior como ensino médio.
