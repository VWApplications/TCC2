\chapter{Considerações finais}

Este trabalho abordou uma metodologia que vem sendo bastante empregada em várias instituições de ensino para preparar melhor os alunos para o mercado de trabalho chamada \textit{Team Based Learning}. Entretanto, as ferramentas encontradas para apoio à sua aplicação não cobriam determinadas áreas críticas necessárias à aplicação do TBL como: apostas em avaliaçôes, o calculo das notas derivadas dessas apostas, feedbacks em tempo real e etc. Elas apenas ajudam a implantação da mesma.

Com o intuito de fechar essa lacuna em relação a ferramenta foi proposto um software chamado PGTBL que irá automatizar todo o processo da metodologia para que o professor possa apenas focar no conteúdo e facilitar o gerenciamento da turma, não tendo que gastar recursos para poder implantar a metodologia em sala de aula.

O \textit{software} foi pensado sempre com foco na qualidade do mesmo, e em aspectos de engenharia de software como engenharia de requisitos, UX, devops, métodos ágeis, medição e análise, testes automatizados, integração e entrega contínua, boas práticas de programação entre outros.

Em relação aos objetivos do TCC, todos foram concluidos com sucesso, desde o estudo do contexto do TBL, elicitação dos
requisitos do software, elaboração do processo de desenvolvimento, execução e aplicação do software em uma disciplina da
UnB aplicando o questionário de satisfação em uso para coletar feedbacks sobre o software.

Os alunos gostaram bastante da ferramenta, apesar da dificuldade de aceitação no início, já que é algo relacionado a
avaliação e menção deles na disciplina, os resultados do questionário de usabilidade foram bem satisfatórios e tornaram
possível a avaliação dos requisitos não funcionais da ferramenta, como: desempenho, suportabilidade, usabilidade,
segurança e confiabilidade. O professor também gostou bastante do resultado, principalmente na facilidade do cálculo das
notas e feedbacks de questões que geraram mais dificuldades nos alunos.

Foi proposto um escopo bem fechado em relação a ferramenta nessas duas entregas planejadas, porém, para trabalhos
futuros, aplicação poderá ter um sistema de \textit{machine learning}, a fase de preparação terá ebooks dinâmicos, e um
sistema de gamificação além de ser extensível a qualquer disciplina ou curso tanto superior como ensino médio.
Também será modificado a arquitetura da aplicação para que ela possa ser extensivel para aplicativos mobile e desktop,
tendo como foco avaliações offline, sem a dependência de internet.
